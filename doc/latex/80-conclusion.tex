\Conclusion

В рамках дипломного проекта был проведён обзор последних тенденций на рынке Web приложений, который
показал востребованность технологии RIA, в частности Flex приложений. Были сформулирваны основные проблемы,
возникающие при тестировании взаимодействия Flex приложений с сервером, на основании которых
поставлено техническое задание. Распространённость технологии
Flex и неотъемлимость фазы тестирования в жизненном цикле программного обеспечения подтверждают
актуальность темы дипломного проекта.

Анализ уже существующих решений по тестированию Flex приложений показал,что ни одно из них в полной мере не
удовлетворяет свормулированным техническим требованиям, в результате чего было принято решение о разработке
собственного программного обеспечения, способного решить поставленную задачу.

Результатом реализации задачи дипломного проекта является программный модуль для тестового фреймворка
Apache JMeter, предоставляющий возможность функционального и нагрузочного тестирования взаимодействия Flex
приложения с сервером через AMF. В качестве серверной технологии используется BlazeDS.

Были проведены работы по модульному и функциональному тестированию разработанного ПО, показавшие его соответствие
заявленным требованиям.

В завершении дипломного проекта было составлено экономическое обоснование и приведен раздел,
посвященный защите интеллектуальной собственности.

Планируется дальнейшее развитие программного модуля с целью улучшения визуализации результатов тестов,
доработки пользовательского интерфейса и поддержки других функций BlazeDS.

