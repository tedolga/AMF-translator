\documentclass[utf8x, 14pt]{G7-32} %размер бумаги устанавливаем А4, шрифт 14 пунктов

% Остальные стандартные настройки убраны в preamble-std.tex
\include{preamble-std}

\begin{document}
\includepdf[pagecommand=\thispagestyle{empty}]{import/title} %вставляю pdf титульник, сгенерированный из вордовского файла

\frontmatter % выключает нумерацию ВСЕГО; здесь начинаются ненумерованные главы: реферат, введение, глоссарий, сокращения и прочее

% Команды \breakingbeforechapters и \nonbreakingbeforechapters
% управляют разрывом страницы перед главами.
% По-умолчанию страница разрывается.

% \nobreakingbeforechapters
% \breakingbeforechapters

% Также можно использовать \Referat, как в оригинале
\begin{abstract}
В этой работе ...
\end{abstract}


\tableofcontents

\Defines
\begin{description}
\item[Распределённый] Слово, которое нельзя употреблять. Но надо протестировать длинные строки в глоссарии.
\end{description}

\Abbreviations %% Список обозначений и сокращений в тексте
\begin{description}
\item[RIA] Rich Internet application, приложения, доступные через сеть Интернет, обладающие особенностями и
функциональностью традиционных настольных приложений.
\item[AMF] Action Message Format, бинарный формат обмена данными.
\item[ПО] программное обеспечение.
\item[GUI] Graphical user interface, графический интерфейс пользователя.
\end{description}


\mainmatter % это включает нумерацию глав и секций в документе ниже

\input{12-intro}% это введение

\chapter{Анализ существующих решений}

В этом разделе выполнены следующие работы: анализ существующих решений в области
автоматизации тестирования, сравнительная оценка вариантов возможных решений, выбор
наиболее оптимального средства автоматизации.

\section{Формулировка критериев выбора}
 
На сегодняшний день существует ряд различных программных комплексов, предоставляющих
возможность для проведения функционального, нагрузочного, регрессионного 
тестирования Flex приложений, поэтому для решения поставленной на дипломный
проект задачи нет необходимсти создавать тестовую утилиту с нуля, разумнее 
будет выбрать одно из существующих решений, и,в случае, если оно не полностью 
удовлетворяет нашим условиям, доработать его. Чтобы выбрать из предложенного 
разнообразия подходящий продукт, сформулируем ряд требований, 
которым должна удовлетворять выбранная нами тестовая утилита.

\begin{enumerate}
\item Возможность проведения нагрузочных тестов. Так как большинство 
web-приложений являются многопользовательскими, проведение нагрузочных
тестов является обязательным условием качественного тестирования. Поскольку
запуск тестов в несколько потоков является довольно сложной задачей, 
инструмент для автоматизации должен обеспечивать удобный и надёжный способ 
многопоточного прогона тестовых сценариев.
\item Поддержка протокола AMF. Утилита должна осуществлять тестирование
взаимодействия Flex приложения с сервером, предоставляя пользователю возможность 
наблюдать трафик. Тестирование пользовательского интерфейса для нашей 
задачи нецелесообразно.
\item Простота создания тестовых сценариев. Утилита должна предоставлять 
пользователю удобных механизм записи и редактирования тестов.
такими как maven и ant, и серверами интеграционного тестирования. Это условие
\item Возможность запуска тестов из командной строки. Данная возможность позволяет
выполнять тесты в автоматическом режиме в системах непрерывной интеграции, что
является необходимостью, особенно в тех случаях, когда над разными частями
тестируемой системы разработчики трудятся независимо и необходимо 
выполнении частых автоматизированных сборок проекта для скорейшего 
выявления и решения интеграционных проблем. Часто производители тестовых 
фреймворков предоставляют свои продукты для проведения интеграционного тестирования, однако 
зачастую они узко заточены под конкретный круг задач, поэтому нас будет интересовать 
возможность запуска тестов на таких инструментах интеграции, как Jenkins и  
Hudson, которые на данный момент широко распротранены и имеют множество плагинов, позволяющих
значительно расширить их существующую функциональность.    
\item Кроссплатформенность. Выбранная нами тестовая утилита должна работать
под управлением различных операционных систем.
\item Условия распростронения продукта. В идеале программное обеспечение должно быть
бесплатным и иметь открытый исходный код с возможность создания собственных расширений,
это является одним из основных критериев отбора, так как тестировщики и разработчики
всегда должны иметь возможность доработки и усовершенствования существующего функционала
тестового фреймворка, чтобы адаптировать его под специфику работы тестируемого ими приложения.
\item Документация. Так как современные инструменты тестирования не менее сложны, чем среды
разработки, большим плюсом будет наличие подробной пользовательской документации,
документации для разработчиков, различных форумов и сайтов, посвящённых данному инструменту
тестирования.
\end{enumerate}

\section{Обзор утилит для тестирования Flex приложений}

\subsection{HP QuickTest Professional}

HP QuickTest Professional (QTP) — один из инструментов автоматизации 
функционального тестирования, является флагманским продуктом компании 
HP в своей линейке. Для разработки автоматизированных тестов QTP 
использует язык VBScript(Visual Basic Scripting Edition ) — скриптовый 
язык программирования, интерпретируемый компонентом Windows Script Host.
Он широко используется при создании скриптов в операционных системах
семейства Microsoft Windows. QTP поддерживает ряд технологий, среди
которых есть и Macromedia Flex. Поддержка Flex осуществляется засчёт
установки плагина, предоставляемого компанией Adobe (Flex QTP add-in).

Чтобы приступить к тестированию приложение необходимо скомпилировать, 
создав для него HTML-оболочку, и развернуть его либо локально, либо на 
web-сервере. Создание тестов в QTP осуществляется следующим образом:
приложение открывается в браузере и все действия, совершаемые пользователем с 
пользовательским интерфейсом приложения записываются в виде строчек Visual Basic скрипта. 
QTP поддерживает запись большинства наиболее часто используемых во Flex 
приложениях событий (операций), связанных с пользовательским интерфейсом. 
Однако часть из них, например некоторые атомарные операции, игнорируются 
во время записи тестов. Пользователь имеет возможность добавить их в 
текст скрипта вручную. Для проверки правильности выполнения теста 
пользователь задаёт ожидаемые значения для выполняемых операций ---
checkpoints. Во время автоматического прогона тестов запускать браузер 
уже не нужно, достаточно лишь указать HTML страницу, используемую для 
тестов. Для выявления причины возникновения ошибок в тестах или каких-либо 
других неполадок можно обратиться к логам Flash Player или настроить и 
включить логирование в QTP. Возможность прогона тестов в несколько потоков
в QTP отсутствует

Для проведения тестов с HP QuickTest Professional необходимо использовать
 браузер Internet Explorer версии 6 или выше. HP QuickTest Professional 
работает с операционными системами семейства Windows и является платным 
программным обеспечением.

\subsection{IBM Rational Functional Tester}

IBM Rational Functional Tester является средством автоматизированного 
регрессивного тестирования, предназначенным для тестирования Java, .NET, 
Web-приложений, включая Flex-приложения, и терминальных приложений на 
платформах Windows и Linux. Functional Tester поддерживает два языка 
сценариев: Java и Visual Basic.NET. Для тестирования Java-приложений в 
программу Functional Tester включена открытая среда разработки Eclipse. 
Установка дополнительных компонентов не требуется. Если требуется 
использовать язык сценариев Visual Basic.NET, перед установкой IBM 
Rational Functional Tester необходимо установить Visual Studio.NET.

Перед началом тестирования приложение загружается в браузере, затем все 
действия,совершаемые пользователем записываются в виде Java или Visual 
Basic.NET скрипта (в зависимости от настроек). Пользователь может 
редактировать скрипты, а также задавать ожидаемые результаты выполнения 
той или иной операции для проверки работы приложения. Как и QTP, IBM 
Rational Functional Tester во время записи тестов добавляет компоненты 
приложения в свой набор объектов. Далее тесты можно запускать в 
автоматическом режиме. Также в IBM Rational Functional Tester имеется 
возможность многократного запуска тестов с различным набором данных. 
Если стандартного набора объектов(таких как кнопки, текстовые поля и т.д.) 
недостаточно, существует возможность самостоятельно добавить во фреймворк 
необходимые объекты. Это возможно благодаря Rational Functional Tester 
proxy software development kit (SDK), который имеет удобное API и 
довольно подробную сопроводительную документацию.
Также IBM Rational Functional Tester упрощает механизм регрессионного 
тестирования. Functional Tester использует усовершенствованную технологию 
ScriptAssure для того, чтобы «изучить» контрольные характеристики 
пользовательского интерфейса, что позволяет идентифицировать те же самые 
средства управления в новой версии, несмотря на внесенные изменения. 
Эти характеристики сохраняются в объектной карте, совместный доступ к 
которой могут получить различные скрипты и участники проекта. 
Благодаря этой карте изменения, внесенные в характеристики распознавания 
объекта, будут отражены во всех скриптах тестирования, что существенно 
упрощает обслуживание. 

Rational Functional Tester поддерживает операционные системы Windows 
2000, XP, Vista, 7, Linux. Является платным программным обеспечением.

\subsection{NeoLoad}

NeoLoad --- профессиональный инструмент нагрузочного тестирования веб-приложений,
в том числе и Flex, работающий на многих платформах, в том числе Windows, Solaris, Linux.
Осуществляя моделирование большого числа пользователей, которые обращаются 
к приложению, NeoLoad делает проверку надежности и производительности 
приложения при разных нагрузках.

Тестирование Flex приложений осуществляется засчёт записи AMF трафика во 
время взаимодействия пользователя с приложением. Все перехваченные запросы 
отображаются в xml формате, что позволяет редактировать содержащиеся в них 
данные. Далее записанные тесты могут быть запущены в несколько потоков.

NeoLoad является платным программным обеспечением с закрытым исходным кодом. 
 
\subsection{Apache JMeter}

JMeter --- инструмент для проведения нагрузочного тестирования, изначально
разрабатывался как средство тестирования web-приложений, в настоящее время 
он способен проводить нагрузочные тесты для JDBC-соединений, FTP, LDAP, SOAP, 
JMS, POP3, IMAP, HTTP и TCP. 
В JMeter реализованы возможность создания большого количества запросов с 
помощью нескольких компьютеров при управлении этим процессом с одного из них,
логирование результатов тестов и разнообразная визуализация результатов в
виде диаграмм, таблиц и т. п.

Ключевое понятие в JMeter –-- план тестирования. План тестирования приложения
представляет собой описание последовательности шагов, которые
будет исполнять JMeter. План тестирования может содержать:

\begin{enumerate}
\item Группы потоков (Thread Groups) --- элемент, позволяющий конфигурировать
многопоточный запуск тестов
\item Контроллеры --- позволяют создаватьтесты со сложной логической структурой.
\item Слушатели --- отображают результаты выполнения тестов
\item Соответствия --- позволяют сравнивать полученные результаты с
ожиданиями пользователя
\item Сэмплеры --- основные элемнты тест-плана, в которых формируется тело
запроса, тестовый шаг.
\end{enumerate}

На данный момент в JMeter нет отдельных средств для тестирования Flex приложений, 
однако есть возможность записи http запросов, содержащих в себе тело AMF 
сообщения, с помощью прокси-сервера. Далее перехваченные запросы могут быть 
перенесены в план тестирования и запущены в режиме нагрузочного тестирования. 
Созданные в JMeter тесты могут быть запущены с помощью систем сборок проектов Maven и 
Ant. Также JMeter свободно интегрируется со многими серверами сборок, такими как Jenkins 
и Hudson.

JMeter является бесплатным кросс-платформенным Java-приложением с открытым
исходным кодом и удобным API. Существует большое количество плагинов к JMeter, 
значительно расщиряющих его базовую функциональность.

\section{Итоги}

В предыдущем разделе был рассмотрен ряд инструментов для тестировани Flex приложений.
Для наглядности и простоты анализа все исследованные продукты были собраны в таблицу и 
охарактеризованы по следующему набору критериев:

\begin{enumerate}
\item Возможность проведения нагрузочных тестов; 
\item Поддержка протокола AMF.
\item Простота создания тестовых сценариев.
\item Возможность запуска тестов из командной строки.
\item Кроссплатформенность;
\item Условия распростронения продукта;
\item Документация.
\end{enumerate}

Для каждого критерия (в таблице указаны их номера), была дана оценка по пятибальной шкале, 
характеризующая степень того, в какой мере тот или иной продукт удовлетворяет данному критерию.

\begin{table}[ht]
\caption{Таблица сравнительного анализа тестовых утилит}
\begin{tabular}{|c|c|c|c|c|c|c|c|}
\hline 
Инструмент тестирования & 1 & 2 & 3 & 4 & 5 & 6 & 7\\
\hline 
HP QuickTest Professional & 2 & 0 & 3 & 4 & 0 & 1 & 4\\
\hline 
IBM Rational Functional Tester & 2 & 0 & 4 & 3 & 5 & 1 & 4\\
\hline 
NeoLoad & 5 & 5 & 4 & 4 & 5 & 1 & 3\\
\hline 
Apache JMeter & 5 & 3 & 4 & 5 & 5 & 5 & 4\\
\hline
\end{tabular} 
\label{tab:tabular}
\end{table}

Из таблицы видно, что по обозначенным критериям наиболее подходящим для 
решения поставленной задачи является инструмент тестирования Apache JMeter, решающим
фактором стало условие распространения продукта с открытым исходным кодом, а также его популярность.
Для того, чтобы JMeter мог обеспечивать полную поддержку тестирования Flex приложений, необходимо
будет расширить его фунционал, добавив отображение AMF сообщений в понятной пользователю 
форме и возможность запуска тестов в несколько потоков без дублирования сессий с сервером.
\chapter{Программная реализация модуля}

\section{Выбор технических средств решения задачи}

 Как показали результаты обзора существующих тестовых фреймворков, на данный момент нет решения польностью
 удовлетворяющего поставленным требованиям. Однако многие рассмотренные программные комплексы уже предоставляют часть
 необходимого нам функционала, поэтому для решения поставленной на дипломный
 проект задачи нет необходимсти создавать тестовую утилиту с нуля, разумнее
 будет выбрать одно из существующих решений, и доработать его.
 Чтобы выбрать из предложенного разнообразия подходящий продукт, составим таблицу, где укажем, в какой степени каждый
  тестовый фреймворк удовлетворяет поставленным требованиям.

  ТУТ ТАБЛИЦА

  Анализ показал, что оптимальным выбором, в наибольшей степени удовлетворяющим поставленным требованиям, является
  Apache JMeter. Именно на базе его функционала будет реализовано решение задачи дипломногот проекта.

  Разработка модуля будет осуществляться с использованием языка Java, что обусловлено одним из требований технического
  задания - кросплатформенность приложения. Программы на Java транслируются в байт-код, выполняемый виртуальной машиной
   Java (JVM) — программой, обрабатывающей байтовый код и передающей инструкции оборудованию как интерпретатор, что
   обеспечивает полную независимости байт-кода от операционной системы и оборудования и позволяет выполнять
   Java-приложения на любом устройстве, для которого существует соответствующая виртуальная машина. Также выбранный нами
   тестовый фреймворк Apache JMeter является стопроцентным Java приложением, поэтому для интеграции с ним целесообразно
   использовать именно этот язык.

На текущий момент основным средством обеспечения взаимодействия Flex клиентов с Java приложениями является технология
BlazeDS --- серверная Java-технология для передачи данных, поддерживающая AMF протокол. BlazeDS является бесплатным
приложением с открытым исходным кодом,разработано компанией Adobe. В силу распространённости BlazeDS, одно из основных
функциональных требований к реализации модуля---обработка AMF сообщений--- будет решаться с помощью средств именно
этой техноогии.

В качестве инструмента автоматизации сборки проектов был выбран Apache Maven  фреймворк для автоматизации сборки
проектов, специфицированных на XML-языке POM (Project Object Model).Основныvb преимуществами Maven являются:

\begin{enumerate}
\item Независимость от OS. Сборка проекта происходит в любой операционной системе. Файл проекта один и тот же.
\item Управление зависимостями. Редко какие проекты пишутся без использования сторонних библиотек(зависимостей), которые
 зачастую тоже в свою очередь используют библиотеки разных версий. Мавен позволяет управлять такими сложными
 зависимостями, что позволяет разрешать конфликты версий и в случае необходимости легко переходить на новые версии
 библиотек.
\item Возможна сборка из командной строки. Такое часто необходимо для автоматической сборки проекта на сервере
(Continuous Integration).
\item Хорошая интеграция с средами разработки. Основные среды разработки на java легко открывают проекты которые
собираются c помощью maven. При этом зачастую проект настраивать не нужно - он сразу готов к дальнейшей разработке.
Как следствие - если с проектом работают в разных средах разработки, то maven удобный способ хранения настроек.
Настроечный файл среды разработки и для сборки один и тот же - меньше дублирования данных и соответственно ошибок.
\item Декларативное описание проекта. B файлах проекта pom.xml содержится его декларативное описание, а не отдельные
команды.
\end{enumerate}

Эффективность разработки программного обеспечения в любом современном проекте подразумевает возможность вести
разработку параллельно с другими участниками проекта.Для оптимизации совместной работы над дипломным проектом было
принято решение о размещении всех файлов проекта в репозитории системы контроля версий Git. Git—это быстрая,
масштабируемая, распределенная система управления версиями с большим набором команд, которые обеспечивают как
операции верхнего уровня, так и полный доступ к внутренним механизмам.

В качестве среды разработки бал выбрана IntellijIdea. http://www.ozon.ru/context/detail/id/2331312/

\section{Реализация поддержки AMF протокола}

\subsection{Общие сведения о BlazeDS}
BlazeDS -серверная Java-технология для передачи данных. Предоставляет ряд сервисов, которые позволяют приложениям 
клиента взаимодействовать с сервером, а также осуществяет передачу данных между несколькими клиентами, подключенными 
к серверу BlazeDS, в режиме реального времени.BlazeDS приложение состоит из двух частей: клиентского приложения и серверного 
J2EE web-приложения. Данная архитектура представлена на следующем рисунке

ТУТ РИСУНОК

Клиентское приложение BlazeDS обычно представляет собой Adobe Flex или AIR приложение. В его состав входят: 

\begin{enumerate}
\item Пользовательский интерфейс приложения. Создаётся с помощью Flex SDK;
\item Один или несколько компонентов BlazeDS: 
\begin{enumerate}
\item RemoteObject --- компонент, предоставляющий клиентскому приложению доступ к методам Java-объектов на стороне сервера; 
\item HTTPService --- компонент, позволяющий клиентскому приложению с помощью http запросов взаимодействовать с JSP, сервлетами, 
ASP страницами через сервер BlazeDS;
\item WebService --- компонент, предназанченный для взаимодействия с веб-сервисами;
\item Producer --- компонент-отправитель сообщений, предназанчен для взаимодействия с сервером сообщений;
\item Consumer --- компонент-получатель сообщений, предназанчен для взаимодействия с сервером сообщений.
\end{enumerate}
\item Набор каналов. На стороне клиента определяются каналы,которые инкапсулируют соединение между Flex клиентом и сервером 
BlazeDS. Для клиентского приложения задаётся набор каналов, упорядоченных по предпочтению. Flex компонент пытается подключиться 
по первому каналу, указанному в списке, и в случае неудачи выбирает следующий канал и т.д, до тех пор пока соединение не будет 
установлено, либо список каналов не кончится.Flex клиенты могут использовать различные типы каналов, такие как AMFChannel и 
HTTPChannel. AMFChannel использует бинарный AMF протокол, а HTTPChannel - небинарный формат AMFX (AMF, преобразованный в XML). 
Выбор канала зависит от ряда факторов, например от типа создаваемого приложения, формата передачи данных, требуемого размера 
сообщений.
\end{enumerate}

Архитектура сервера BlazeDS представлена на ...

Тут  РИСУНОК

BlazeDS сервер базируется на технологии J2EE. Взаимодействие клиента и сервера BlazeDS происхоит следующим 
образом : Flex клиент посылает запрос по определённому каналу, далее запрос направляется в соответствующий каналу 
компонент endpoint, который является точкой обработки получаемых сервеом сообщений различного типа. Затем сообщение 
декодируется и проходит через ряд Java-объектов - MessageBroker , Service object, Destination object и Adapter object.
 Adapter object либо обрабатывает запрос локально, либо связывается с какой-либо backend системой или 
 удалённым сервером. После запроса происходит обратный процесс.

\subsection{Сериализация и десериализация сообщений}

\subsection{AMF клиент}

В BlazeDS существует механизм, Java AMF Client, позволяющий совершать удалённые вызовы методов и обрабатывать ответы
сервера. Преимущество использования Java AMF Client заключается в том, что сериализация и десериализация AMF сообщений,
отправляемых клиентом и сервером, а также установка http соединения, полностью обеспечивается данной технологией.

Классы, реализующие функционал Java AMF Client, находятся в пакете flex.messaging.io.amf.client. Основным классом
является AMFConnection. Пример его работы показан ...

 ТУТ ПРИМЕР КОДА

AMFConnection устанавливает соединение с удалённым объектом по указанному URL с помощью метода connect().
В случае успешной установки соединения метод call() отправляет AMF запрос пользователю, в качестве параметров метод
принимает имя вызываемого на стороне сервера метода и его параметры, представленные в виде массива объектов.

В данном примере возможно появление двух видов исключительных ситуаций:

\begin{enumerate}
\item ServerStatusException --- в случае появления сообщения об ошибке от сервера.
\item ClientStatusException --- в случае ошибки установки соединения с сервером или при непредвиденном разрыве
соединения.
\end{enumerate}

\section{Интеграция с JMeter}

\subsection{Общая архитектура JMeter}

Структура проекта JMeter изображена на ...

ТУТ РИСУНОК

\begin{enumerate}
\item bin --- содержит в себе .bat и .sh файлы для запуска JMeter, файл ApacheJMeter.jar и файлы настроек;
\item docs --- директория, содержащая документацию по проекту;
\item extras --- дополнительне фалйлы для утилиты ant;
\item lib --- jar файлы библиотек, используемых в JMeter;
\item lib/ext --- jar файлы ядра и отдельных компонентов JMeter;
\item src --- исходные коды JMeter;
\item test --- юнит-тесты;
\item xdocs --- xml файлы для документации (JMeter генерирует документацию из xml).
\end{enumerate}

Рассмотрим подробнее содержимое директории src:

\begin{enumerate}
\item component --- директория, содержащая общие для различных протоколов элементы, такие как визуалайзеры,
соответствия и т.д;
\item core --- ядро JMeter, содержит базовые интерфейсы и абстрактные классы;
\item examples --- примеры, демонстрирующие использование компонентов фреймворка;
\item functions --- стандартные функции, используемые всеми компонентами;
\item jorphan --- утилитные классы;
\item monitor --- элементы мониторинга сервера Tomcat 5;
\item protocol --- содержит реализации компонентов Jmeter для различных протоколов.
\end{enumerate}

В архитектуре JMeter ядро, содержащее в себе интерфейсы и абстрактные классы,а также базовый функционал,
отделёны от конкретных реализаций компонентов для различных протоколов. Это сделано для того, чтобы разработчики
могли добавлять поддержку новых протоколов без сборки всего приложения. Таким образом, для того, чтобы добавить
в JMeter элементы тестирования Flex приложений, нужно будет переопределить несколько базовых
компонентов JMeter, собрать jar файл модуля, и поместить его в директорию lib/ext --- новый функционал будет
автоматически подхвачен JMeter.

Прежде чем приступить к созданию различных компонентов, опишем ряд общих правил их реализации,
которые необходимы для того, чтобы элемент правильно работал в среде JMeter. В основном это относится к
графическому интерфейсу пользователя (GUI).

В JMeter код GUI элемента отделён от функционального кода элемента, поэтому реализуя новый компонент следует
создавать отдельные классы для рабочего функционала и графического представления. GUI элемент, в зависимости
 от его предназанчения, должен расширять один из представленных ниже абстрактных классов.

\begin{enumerate}
\item AbstractSamplerGui
\item AbstractAssertionGui
\item AbstractConfigGui
\item AbstractControllerGui
\item AbstractPostProcessorGui
\item AbstractPreProcessorGui
\item AbstractVisualizer
\item AbstractTimerGui
\end{enumerate}

Следующим шагом является реализация метода getResourceLabel().Этот метод должен возвращать имя ресурса,
представляющего компонент.

Чтобы GUI создаваемого вами тестового элемента соответсвовало стилю JMeter, следует добавить в него стандартную
рамку JMeter, которая создаётся следующим образом --- setBorder(makeBorder()). Также необходимо создать
панель с именем элемента методом makeTitlePanel(). Обычно её располагают в верхней центральной части панели
элемента.

Важным пунктом является реализация метода public void configure(TestElement el), который отвечает за отображение
параметров тестового элемента в GUI. Первой строчкой в методе должен быть вызов super.configure(e), что обеспечит
выполнение некоторых стандартных действий, как например отображение имени элемента. Обратное действие ---
передача данных из GUI в тестовый элемент --- обеспечивается методом public void modifyTestElement(TestElement e),
который также нужно реализовать. В имплиментацию этого метода тоже включается строчка
super.configureTestElement(e).

Другой необходимый в создании GUI метод --- public TestElement createTestElement(). Он должен новый экземпляр
тестового элемента и передавать его в метод modifyTestElement(TestElement) в качестве параметра.

Все вышеобозначенные правила разграничения GUI и функционала элемента облегчают разработчикам реализацию графического
интерфейса. Хотя программист и не освобождается от процедуры создания и размещения компонентов графического
интерфейса, всё взаимодействие GUI и соответствующего ему тестового элемента обеспечивается JMeter.

\subsection{Реализация модуля отправки AMF сообщений}



\subsection{Реализация прокси-сервера}

\section{Руководство пользователя}

\subsection{Установка JMeter}

Для начала необходимо скачать с сайта производителя zip архив, содержащий все необходимы для установки файлы,
а затем распаковать его на диске. Место установки JMeter далее будем называть JMETER\_HOME.

\subsection{Установка модуля amf-translator}

Чтобы подключить к JMeter модуль amf-translator, достаточно добавить jar-файл приложения в каталог JMETER\_HOME/lib/ext.

\subsection{Запуск приложения}

Приложение запускается с помощью файла jmeter.bat или ApacheJMeter.jar, находящихся в каталоге JMETER\_HOME/bin.

\subsection{Настройка прокси-сервера}

После запуска в окне приложения с левой стороны нам доступно дерево элементов. Чтобы создать прокси сервер
с поддержкой протокола AMF, необходимо правой кнопкой мыши кликнуть по элементу WorkBench,
а затем добавить элемент AMF Proxy Server (WorkBench > Add > Non-Test Elements > AMF Proxy Server).

\begin{figure}[ht]
\center{\includegraphics[height=120mm, width=160mm]{fig/development/proxySettings.png}}
\caption{Настройка прокси-сервера}
\label{ris:proxySettings.png}
\end{figure}

В поле AmfProxy Port необходимо указать номер порта, который будет слушать наш прокси сервер.
Если указать, например, 9090, то прокси-сервер будет запущен на localhost:9090.
Затем точно такие же настройки прокси-сервера устанавливаются в браузере, с помощью которого будет
производиться тестирование. Также стоит убедиться, что указанный Вами порт уже не занят другим приложением.

\subsection{Запись тестового сценария}

После того, как в AMF Proxy Server установлены все необходимые параметры, нажимается кнопка Start,
запускающая прокси-сервер.

\begin{figure}[h]
\center{\includegraphics[height=120mm, width=160mm]{fig/development/proxyStart.png}}
\caption{Запуск прокси-сервера}
\label{ris:proxyStart.png}
\end{figure}

Затем тестируемое приложение открывается в браузере, для которого также
применены соответсвующие настройки прокси-сервера, и пользователь может выполнять с Flex приложением
необходимые операции, которые будут записываться AMF Proxy Server в виде элементов AMF RPC Sampler и в
дальнейшем могут быть перенесены в тест-план. Чтобы завершить запись тестовых запросов, необходимо
нажать кнопку Stop. После завершения записи тестов, все перехваченные запросы отображаются в дереве
элементов JMeter в качестве дочерних элементов AMF Proxy Server.

\subsection{Создание тест-плана}

Создание тест-плана в JMeter осуществляется следующим образом.
В первую очередь добавляется группа потоков - Thread Group (Test Plan > Threads (Users) > Thread Group).

\begin{figure}[ht]
\center{\includegraphics[height=120mm, width=160mm]{fig/development/testplan.png}}
\caption{Создание тест-плана}
\label{ris:testplan.png}
\end{figure}

Этот элемент является ключевым в тест-плане JMeter, именно его функционал отвечает за реализацию
нагрузочного тестирования --- многопоточного запуска последовательности тестовых шагов.
В данном элементе задается следующее:

\begin{enumerate}
\item Действия, которые будут производиться в случае, если в тест выполняется с ошибкой
(Action to be taken after a Sampler error);
\item Число потоков, в которое будут запускаться шаги тест-плана (Number of Threads);
\item Интервал, в течение которого будет запущено указанное в предыдущем параметре
число потоков (Ramp-Up Period);
\item Число повторений набора тестов (Loop Count);
\item Расписание запуска тестов (Scheduler).
\end{enumerate}

Затем в Thread Group в качестве дочерних элементов добавляются шаги тестов, которые
будут запускаться с указанными характеристиками. В нашем случае мы переносим элементы
AMF RPC Sampler, записанные с помощью прокси.

AMF RPC Sampler представляет собой форму для вызова процедур java-объектов на стороне
сервера. Чтобы вызвать метод удалённого объекта, необходимо заполнить следующие
поля: 

\begin{enumerate}
\item Endpoint Url - URL, по которому отправляется запрос
\item AMF Call - имя удалённого объекта и процедуы, которая должна быть вызвана; 
Например, если мы хоти вызвать у объекта registrationDestionation метод registerUser, 
в этом поле следует написать registrationDestination.registerUser;
\item Request Parameters - параметры метода, вызываемого у объекта (если они 
существуют).
\end{enumerate}

На Рис.~\ref{ris:amfSampler.png} представлен интерфейс AMF RPC Sampler.
\begin{figure}[ht]
\center{\includegraphics[height=120mm, width=160mm]{fig/development/amfSampler.png}}
\caption{Интерфейс элемента AMF RPC Sampler}
\label{ris:amfSampler.png}
\end{figure}
 
Помимо сэмплера в тест-план также следует добавить визуалайзер результатов, чтобы иметь возможность
отслеживать ход тестового сценария (Thread Group > Add > Listener ).
JMeter предлагает большой выбор таких элементов, для примера будем использовать View Results Tree.
После того как план сформирован, он может быть сохранён. (File > Save Test Plan As...)

\subsection{Запуск тестов}

Чтобы запустить содержимое элемента Test Plan, необходимо выбрать в основном меню Run > Start.
После заврешения прогона тестов результаты их выполнения можно наблюдать в View Results Tree.

\begin{figure}[ht]
\center{\includegraphics[height=110mm, width=160mm]{fig/development/testResults.png}}
\caption{Отображение результатов тестов в визуалайзере View Results Tree}
\label{ris:testResults.png}
\end{figure}

В правой части визуалайзера отображается дерево элементов тест плана и статус 
их выполнения - если элемент подсвечен зелёным цветом, то шаг выполнен 
успешно. В случае с AMF RPC Sampler это значит, что соединение с сервером было установлено 
и вызов метода удалённого объекта прошёл без ошибок. Иначе тест не считантся пройденным и 
элемент подсвечивается красным цветом. Также на вкладках слева предоставляется 
возможность просмотра отправленного запроса и полученного от сервера ответа.  

\section{Методика тестирования}
\chapter{Методика тестирования}

Одной из задач дипломного проекта является создание методики тестирования взаимодействия Flex приложения
с сервером. Методика должна в качестве основного инструмента тестирования использовать разработанный программный модуль.

Демонстрация методики осуществляется на примере тестирования Flex клиента, обладающего
несложным функционалом: клиент отправляет серверу запрос с указанными пользователем параметрами, а сервер отвечает
сообщением, в котором содержится переданная ему информация. На Рис.~\ref{ris:flexClient.png} изображён web-интерфейс приложения.

\begin{figure}[ht]
\center{\includegraphics[height=70mm, width=100mm]{fig/development/flexClient.png}}
\caption{Web интерфейс Flex приложения}
\label{ris:flexClient.png}
\end{figure}

Пользователь указывает необходимые данные, нажимает кнопку "Send" и в текстовом поле снизу появляется
ответ сервера.

Сформулируем ряд требований к данному приложению.

Функциональные требования:

\begin{enumerate}
\item При отправке корректного запроса сервер должен отвечать сообщением, содержащим в себе следующие
данные введённые пользователем: User, Email, Stay at System/Don't stay at system, Receive letters/Don't
receive letters;
\item При отправке некорректного запроса сервер должен отвечать сообщением об ошибке с соответствующим
описанием.
\end{enumerate}

Требования к нагрузке:

\begin{enumerate}
\item Приложение должно выполнять свой функионал при работе с ним ста пользователей одновременно.
\end{enumerate}

На основании сформулированных функциональных и нагрузочных требований к приложению составлен план тестирования
и набор тестовых случаев.

\section{План тестирования}

\subsection{Объект тестирования}

Объектом тестирования является взаимодействие Flex клиента с сервером BlazeDS.

\subsection{Цели и приоритеты тестирования}

Целью тестирования является проверка соответсвия программного обеспечения функциональным требованиям,
а также работоспособности программного обеспечения под заданной нагрузкой.

\subsection{Стратегии тестирования}

К тестированию прлиложения планируется применение следующих видов тестирования:

\begin{enumerate}
\item функциональное тестирование --- для проверки соответствия требованиям, указанным в техническом задании.
\item нагрузочное тестирование --- для проверки работоспособности ПО в заданных условиях.
\end{enumerate}

\subsection{Последовательность работ}

\begin{enumerate}
\item Проектирование набора тестовых случаев;
\item Подготовка тестового окружения;
\item Проведение тестирования;
\item Анализ результатов;
\end{enumerate}

\subsection{Критерии начала тестирования}

\begin{enumerate}
\item законченность разработки требуемого функционала;
\item покрытие кода тестируемого приложения юнит-тестами;
\item наличие задокументированных требований к программному обеспечению;
\item наличие набора тестовых случаев;
\item готовность тестового окружения.
\end{enumerate}

\subsection{Критерии окончания тестирования}

\begin{enumerate}
\item соответствие продукта функциональным требованиям;
\item стабильность работы продукта под заданной нагрузкой;
\end{enumerate}

\subsection{Тестовое окружение}

Для проведения тестирования необходимо следующее программное обеспечение:

\begin{enumerate}
\item сервер BlazeDS;
\item Flex приложение;
\item web-браузер c установленным Flash Player;
\item Apache JMeter.
\end{enumerate}

\section{Набор тестовых случаев}

\subsection{Настройка тестового окружения}

Перед проведением тестов необходимо настроить тестовую среду следующим образом:

\begin{enumerate}
\item установить программное обеспечение, указанное в разделе "Тестовое окружение" плана приёмочных испытаний.
Место установки JMeter далее будем называть JMETER\_HOME;
\item подключить к JMeter тестируемый модуль amf-translator --- добавить jar-файл приложения в каталог
JMETER\_HOME/lib/ext;
\item развернуть Flex приложение на сервере BlazeDS;
\item запустить JMeter --- приложение запускается с помощью файла jmeter.bat или ApacheJMeter.jar, находящихся в
каталоге JMETER\_HOME/bin.
\end{enumerate}

\subsection{Функциональные тесты}

{\bfseries Проверка обработки корректного запроса приложения}

Предусловия:

\begin{enumerate}
\item Добавить в Test Plan группу потоков - Thread Group (Test Plan > Threads (Users) > Thread Group).

\begin{figure}[ht]
\center{\includegraphics[height=80mm, width=120mm]{fig/development/testplan.png}}
\caption{Создание группы потоков}
\label{ris:testplan.png}
\end{figure}

\item Задать для Thread Group следующие параметры:

\begin{enumerate}
\item Действие, которое будут производиться в случае, если в тест выполняется с ошибкой
(Action to be taken after a Sampler error) --- Continue ;
\item число потоков, в которое будут запускаться шаги тест-плана (Number of Threads) установить равным единице;
\item Интервал, в течение которого будет запущено указанное в предыдущем параметре
число потоков (Ramp-Up Period) установить равным единице;
\item Число повторений набора тестов (Loop Count) установить равным единице;
\end{enumerate}

\item добавить визуалайзер результатов, чтобы иметь возможность отслеживать ход выполнения теста (Thread Group >
Add > Listener > View Results Tree)
\end{enumerate}

Шаги теста:

\begin{enumerate}
\item В Thread Group в качестве дочернего элемента добавить AMF RPC Sampler;
\item В AMF RPC Sampler записать верные параметры AMF запроса, а именно:
\begin{enumerate}
\item Endpoint Url - URL, по которому отправляется запрос;
\item AMF Call - имя удалённого объекта и процедуы(Например, если мы хоти вызвать у объекта registrationDestionation метод registerUser,
в этом поле следует написать registrationDestination.registerUser);
\item Request Parameters - параметры, в том порядке, в котором они должны быть переданы методу.
\end{enumerate}

\begin{figure}[ht]
\center{\includegraphics[height=80mm, width=120mm]{fig/development/amfSampler.png}}
\caption{Интерфейс элемента AMF RPC Sampler}
\label{ris:amfSampler.png}
\end{figure}

\item Запустить содержимое элемента Test Plan (Run > Start);
\end{enumerate}

Ожидаемый результат:

\begin{enumerate}
\item После завершения прогона тестов во View Results Tree в дереве элементов тест плана элемент AMF RPC Sampler
посдвечен зелёным цветом (сервер обработал запрос и не прислал сообщений об ошибке);
\item во вкладке Response Data элемента View Results Tree отображается сообщение сервера, соответствующее
переданным параметрам.
\end{enumerate}

\begin{figure}[ht]
\center{\includegraphics[height=80mm, width=120mm]{fig/development/positiveTest.png}}
\caption{Результаты корректного теста}
\label{ris:positiveTest.png}
\end{figure}

{\bfseries Проверка отправки некорректного AMF запроса}

Предусловия:

\begin{enumerate}
\item Добавить в Test Plan группу потоков - Thread Group (Test Plan > Threads (Users) > Thread Group).
\item Задать для Thread Group следующие параметры:

\begin{enumerate}
\item Действие, которое будут производиться в случае, если в тест выполняется с ошибкой
(Action to be taken after a Sampler error) --- Continue ;
\item число потоков, в которое будут запускаться шаги тест-плана (Number of Threads) установить равным единице;
\item Интервал, в течение которого будет запущено указанное в предыдущем параметре
число потоков (Ramp-Up Period) установить равным единице;
\item Число повторений набора тестов (Loop Count) установить равным единице;
\end{enumerate}

\item добавить визуалайзер результатов, чтобы иметь возможность отслеживать ход выполнения теста (Thread Group >
Add > Listener > View Results Tree)
\end{enumerate}

Шаги теста:

\begin{enumerate}
\item В Thread Group в качестве дочернего элемента добавить AMF RPC Sampler;
\item Ввести в AMF RPC Sampler неверное имя метода удалённого объекта и запустить содержимое элемента Test Plan (Run > Start);
\end{enumerate}

Ожидаемый результат:

\begin{enumerate}
\item После завершения прогона тестов во View Results Tree в дереве элементов тест плана элемент AMF RPC Sampler
подсвечен красным цветом (тест не пройден);
\item Ответ сервера во вкладке View Results Tree должен содержать сообщение о соответсвующей ошибке.

\begin{figure}[ht]
\center{\includegraphics[height=80mm, width=120mm]{fig/development/negativeTest.png}}
\caption{Результаты некорректного теста}
\label{ris:negativeTest.png}
\end{figure}

\end{enumerate}

\subsection{Нагрузочные тесты}

{\bfseries Проверка работоспособности приложения под заданной нагрузкой}

Предусловия:

\begin{enumerate}
\item Добавить элемент AMF Proxy Server (WorkBench > Add > Non-Test Elements > AMF Proxy Server);

\begin{figure}[ht]
\center{\includegraphics[height=80mm, width=120mm]{fig/development/proxySettings.png}}
\caption{Настройка прокси-сервера}
\label{ris:proxySettings.png}
\end{figure}

\item В поле AmfProxy Port необходимо указать номер порта, который будет слушать наш прокси сервер.
Если указать, например, 9090, то прокси-сервер будет запущен на localhost:9090;
\item Запустить браузер и указать там точно такие же настройки прокси-сервера.
Также стоит убедиться, что указанный Вами порт уже не занят другим приложением;
\item Запустить прокси-сервер, нажав кнопку "Start";
\item Далее открыть в браузере Flex приложение, и отправить пять-шесть запросов с корректными параметрами;
\item Завершить запись тестовых запросов, нажав кнопку "Stop";

\begin{figure}[ht]
\center{\includegraphics[height=80mm, width=120mm]{fig/development/proxyRequests.png}}
\caption{Запись запросов}
\label{ris:proxyRequests.png}
\end{figure}

\end{enumerate}

Шаги теста:

\begin{enumerate}
\item Добавить в Test Plan группу потоков - Thread Group (Test Plan > Threads (Users) > Thread Group).
\item Задать для Thread Group следующие параметры:

\begin{enumerate}
\item Действие, которое будут производиться в случае, если в тест выполняется с ошибкой
(Action to be taken after a Sampler error) --- Continue ;
\item число потоков, в которое будут запускаться шаги тест-плана (Number of Threads) установить равным 100 (
согласно требованиям к нагрузке на приложение);
\item Интервал, в течение которого будет запущено указанное в предыдущем параметре
число потоков (Ramp-Up Period) установить равным единице;
\item Число повторений набора тестов (Loop Count) установить равным единице;
\item Задать расписание запуска тестов: время начала тестов и время окончания тестов должны отличаться на
шесть часов, продолжительность теста равна 21600 секундам, задержка --- одной секунде;

\begin{figure}[ht]
\center{\includegraphics[height=80mm, width=120mm]{fig/development/threadParams.png}}
\caption{Настройки Thread Group}
\label{ris:threadParams.png}
\end{figure}

\end{enumerate}

\item перенести записанные с помощью AMF Proxy Server запросы в Test Plan
\item добавить визуалайзер результатов, чтобы иметь возможность отслеживать ход выполнения теста (Thread Group >
Add > Listener > View Results Tree)
\item Запустить тест план.
\end{enumerate}

Ожидаемый результат:

 После завершения прогона тестов во View Results Tree в дереве элементов все тесты должны быть пройдены ---
 корректный ответ от сервера должен быть получен для каждого элемента.

 \begin{figure}[ht]
\center{\includegraphics[height=80mm, width=120mm]{fig/development/load.png}}
\caption{Результаты нагрузочного тестирования}
\label{ris:load.png}
\end{figure}

Разработанная методика демонстрирует варианты использования реализованного программного модуля в
качестве интсрумента тестирования взаимодействия Flex клиента с сервером. В частности в наборе
тестовых случаев были представлены примеры записи тестовых сценариев, отправка AMF запросов и
проверка результатов их выполнения, а также имитация работы с приложением большого числа пользователей
путём многопоточного запуска тестов.
















\chapter{Технико-экономическое обоснование}

\section{Концепция экономического обоснования}

На сегодняшний день одним из наиболее перспективных направлений в разработке
web-приложений является концепция Rich Internet Application (в дальнейшем RIA) --- это приложения,
доступные через сеть Интернет, обладающие особенностями и функциональностью традиционных настольных приложений.
Одной из наиболее распространенных технологий разработки RIA является Adobe Flex.
Flex приложения предоставляют возможность реализации клиент-серверного взаимодействия на основе бинарного формата
обмена данными --- AMF(Action Message Format). AMF более экономичен по трафику по сравнению с XML и позволяет
передавать типизированные объекты.
Как известно огромную роль в жизненном цикле программного обеспечения играет фаза тестирования.
Автоматизированное тестирование является его составной частью. Оно использует программные средства для
выполнения тестов и проверки их результатов, что помогает сократить время тестирования и упростить его процесс,
а также может дать возможность выполнять определенные тестовые задачи намного быстрее и эффективнее чем это может
быть сделано вручную. Однако использование AMF вызывает ряд трудностей для реализации автоматизации
функционального и нагрузочного тестирования взаимодействия сервера и Flex клиента, связанных с
бинарной природой протокола. Так как amf сообщение представляет собой совокупность байтов, разработчикам
и тестировщикам трудно считывать и изменять содержащуюся в нём информацию. Отдельной проблемой является
нагрузочное тестирование таких приложений --- имитация работы с приложением большого количества пользователей
за счёт запуска набора тестов в несколько потоков. Выходом из сложившейся ситуации является разработка
программного обеспечения, в значительной степени снижающего сложность осуществления функционального и
нагрузочного тестирования взаимодействия клиента и сервера по AMF-протоколу, а также разработка
методики функционального и нагрузочного тестирования использованием предложенного ПО.

Целью технико-экономического обоснования является определение экономической целесообразности реализации проекта.
Этапами ТЭО являются:

\begin{enumerate}
\item трудоемкость и календарный план выполнения НИР;
\item смета затрат на проведение НИР;
\item комплексная оценка эффективности НИР.
\end{enumerate}

\section{Трудоемкость и календарный план выполнения НИР}

\begin{landscape}
\begin{longtable}{|c|c|c|c|c|c|c|c|c|c|c|c|c|c|c|c|c|c|c|c|c|c|c|c|c|c|c|c|c|c|}
\caption{Трудоемкость и календарный план выполнения НИР}
\label{tab:longtable}
\\ \hline
\multirow{2}{*}{№}&\multirow{2}{*}{Этапы и работы}&\multicolumn{2}{|c|}{Трудоемкость, чел.-дн.}&\multirow{2}{*}{Численность, чел.}&\multirow{2}{*}{Длительность, дн.}&\multicolumn{24}{|c|}{Продолжительность работы (пятидневка)}\\\cline{3-4}\cline{7-30}
&&Исполнитель&Руководитель НИР&&&4&9&14&19&24&29&33&36&41&46&47&52&53&58&59&64&69&74&78&83&86&91&111&113\\\cline{3-4}\cline{7-30}
\hline \endfirsthead
\subcaption{Продолжение таблицы~\ref{tab:longtable}}
\\ \hline \endhead
\hline \subcaption{Продолжение на след. стр.}
\endfoot
\hline \endlastfoot
1& • & • & • & • & • & • & • & • & • & • & • & • & • & • & • & • & • & • & • & • & • & • & • & • & • & • & • & • & • \\
\hline 
2& • & • & • & • & • & • & • & • & • & • & • & • & • & • & • & • & • & • & • & • & • & • & • & • & • & • & • & • & • \\
\hline 
3 & • & • & • & • & • & • & • & • & • & • & • & • & • & • & • & • & • & • & • & • & • & • & • & • & • & • & • & • & • \\
\hline 
4 & • & • & • & • & • & • & • & • & • & • & • & • & • & • & • & • & • & • & • & • & • & • & • & • & • & • & • & • & • \\
\hline 
5 & • & • & • & • & • & • & • & • & • & • & • & • & • & • & • & • & • & • & • & • & • & • & • & • & • & • & • & • & • \\
\hline 
6 & • & • & • & • & • & • & • & • & • & • & • & • & • & • & • & • & • & • & • & • & • & • & • & • & • & • & • & • & • \\
\hline 
7 & • & • & • & • & • & • & • & • & • & • & • & • & • & • & • & • & • & • & • & • & • & • & • & • & • & • & • & • & • \\
\hline 
8 & • & • & • & • & • & • & • & • & • & • & • & • & • & • & • & • & • & • & • & • & • & • & • & • & • & • & • & • & • \\
\hline 
9 & • & • & • & • & • & • & • & • & • & • & • & • & • & • & • & • & • & • & • & • & • & • & • & • & • & • & • & • & • \\
\hline 
10 & • & • & • & • & • & • & • & • & • & • & • & • & • & • & • & • & • & • & • & • & • & • & • & • & • & • & • & • & • \\
\hline 
11 & • & • & • & • & • & • & • & • & • & • & • & • & • & • & • & • & • & • & • & • & • & • & • & • & • & • & • & • & • \\
\hline 
12 & • & • & • & • & • & • & • & • & • & • & • & • & • & • & • & • & • & • & • & • & • & • & • & • & • & • & • & • & • \\
\hline 
13 & • & • & • & • & • & • & • & • & • & • & • & • & • & • & • & • & • & • & • & • & • & • & • & • & • & • & • & • & • \\
\hline 
14 & • & • & • & • & • & • & • & • & • & • & • & • & • & • & • & • & • & • & • & • & • & • & • & • & • & • & • & • & • \\
\hline 
15 & • & • & • & • & • & • & • & • & • & • & • & • & • & • & • & • & • & • & • & • & • & • & • & • & • & • & • & • & • \\
\hline 
16 & • & • & • & • & • & • & • & • & • & • & • & • & • & • & • & • & • & • & • & • & • & • & • & • & • & • & • & • & • \\
\hline 
16 & • & • & • & • & • & • & • & • & • & • & • & • & • & • & • & • & • & • & • & • & • & • & • & • & • & • & • & • & • \\
\hline
16 & • & • & • & • & • & • & • & • & • & • & • & • & • & • & • & • & • & • & • & • & • & • & • & • & • & • & • & • & • \\
\hline
16 & • & • & • & • & • & • & • & • & • & • & • & • & • & • & • & • & • & • & • & • & • & • & • & • & • & • & • & • & • \\
\hline
16 & • & • & • & • & • & • & • & • & • & • & • & • & • & • & • & • & • & • & • & • & • & • & • & • & • & • & • & • & • \\
\hline
16 & • & • & • & • & • & • & • & • & • & • & • & • & • & • & • & • & • & • & • & • & • & • & • & • & • & • & • & • & • \\
\hline
16 & • & • & • & • & • & • & • & • & • & • & • & • & • & • & • & • & • & • & • & • & • & • & • & • & • & • & • & • & • \\
\hline
\end{longtable}
\end{landscape}

Трудоемкость выполнения работы исполнителем  составляет 106 чел.-дней, а руковолителем 7 чел.-дней.
Общая продолжительность выполнения данной НИР 113 дней (чуть больше 15 недель).

\section{Cмета затрат на проведение НИР}

Проводится расчет затрат, связанных с проведением НИР. Основные статьи калькуляции приведены в табл. 4.1.


\chapter{Охрана интеллектуальной собственности}

\section{Интеллектуальная собственность}

Согласно определению интеллектуальной собственности, принятому в российском законодательстве,
а также на основании определения Стокгольмской конференции от 14 июля 1967 г., компьютерные программы
относятся к объектам интеллектуальной собственности. Компьютерным программам предоставляется охрана
нормами авторского права как литературным произведениям в соответствии с Бернской конвенцией.
В Российской Федерации вопросы предоставления правовой охраны программам регулируются Гражданским
кодексом РФ, Часть~4~(ГК~РФ~Ч.4).

\section{Программа для ЭВМ}

Под программой для ЭВМ понимается <<... представленная в объективной форме совокупность данных и команд,
предназначенных для функционирования ЭВМ и других компьютерных устройств в целях получения определенного
результата>>. Кроме того, в понятие программы для ЭВМ входят <<...подготовительные материалы, полученные
в ходе разработки программы для ЭВМ, и порождаемые ею аудиовизуальные отображения>>~[2,~ст.~1261].
С точки зрения программистов и пользователей программа для ЭВМ представляет собой детализацию алгоритма
решения какой-либо задачи и выражена в форме определенной последовательности предписаний, обеспечивающих
выполнение компьютером преобразования исходных данных в искомый результат.

Можно выделить следующие объективные формы представления программы для ЭВМ:
\begin{enumerate}
\item исходная программа (или исходный текст) --- последовательность предписаний на алгоритмическом языке
высокого уровня, предназначенных для автоматизированного перевода этих предписаний в последовательность команд
в объектном коде;
\item рабочая программа (или объектный код) --- последовательность машинных команд, т. е. команд, представленных
на языке, понятном ЭВМ;
\item программа, временно введенная в память ЭВМ, --- совокупность физических состояний элементов памяти
запоминающего устройства ЭВМ (ОЗУ), сохраняющихся до прекращения подачи электропитания к ЭВМ;
\item программа, постоянно хранимая в памяти ЭВМ, --- представленная на языке машины команда (или серия команд),
выполненная в виде физических особенностей участка интегральной схемы, сохраняющихся независимо от подачи
электропитания.
\end{enumerate}

Исходная и рабочая программы представляются в электронном виде.
Правовая охрана программ для ЭВМ распространяется только в отношении формы их выражения и <<… не распространяется
на идеи, концепции, принципы, методы, процессы, системы, способы, решения технических, организационных или иных
задач, открытия, факты, языки программирования>>~[2,~ст.1259,~п.~5].

\section{Авторское право на программу для ЭВМ}

Предпосылкой охраноспособности программы для ЭВМ и базы данных является их творческий характер, т. е. они
должны быть продуктом личного творчества автора. Творческий характер деятельности автора предполагается до
тех пор, пока не доказано обратное~[2,~ст.~1257].

Момент возникновения авторского права является важнейшим юридическим фактом, который устанавливается в силу
создания программы для ЭВМ. <<Для возникновения, осуществления и защиты авторских прав не требуется регистрация
произведения или соблюдение каких-либо иных формальностей>>~[2,~ст.1259,~п.4].

Закон устанавливает, что обнародование программы не является обязательным условием для возникновения прав на нее:
<<Авторские права распространяется как на обнародованные, так и на необнародованные произведения, выраженные в
какой-либо объективной форме ...>>~[2,~ст.~1259,~п.~3].

Таким образом, только сам факт создания программы, зафиксированной в объективной форме, является основанием
возникновения авторского права на нее.

Каждая составляющая понятия использования программы для ЭВМ имеет конкретное содержание, которое также
определено законом:
\begin{enumerate}
\item воспроизведение --- <<... изготовление одного или более экземпляров произведения или его части
в любой материальной форме, … в том числе запись в память ЭВМ>>~[2,~ст.~1270,~п.~2,~п.п.1];
\item распространение –-- предоставление доступа к произведению <<... путем продажи или иного отчуждения его
оригинала или экземпляров>>~[2,~ст.~1270,~п.~2,~п.п.2];
\item публичный показ --- <<... любая демонстрация оригинала или экземпляров произведения непосредственно ...
либо с помощью технических средств в месте, открытом для свободного посещения, или в месте, где присутствует
значительное число лиц ...>>~[2,~ст.~1270,~п.~2,~п.п.3].
\end{enumerate}

В целях оповещения о своих правах правообладатель <<... вправе использовать знак охраны авторского права,
который помещается на каждом экземпляре произведения и состоит из следующих элементов:
латинской буквы С в окружности; имени или наименования правообладателя; года первого опубликования
произведения>>~[2,~ст.~1271].

Исключительные права на программу переходят по наследству в установленном законом порядке, и их можно
реализовать в течение срока действия авторского права.

Передача прав на материальный носитель не влечет за собой передачи каких-либо прав на программу для
ЭВМ~[2,~ст.1227]. Иными словами, передача носителя информации (например, диска) с зафиксированной на
нем программой третьему лицу не означает передачи каких-либо прав на эту программу.

\section{Правообладание}

Если человек (или группа людей) самостоятельно, по личной инициативе создал программу для ЭВМ,
то он является одновременно и автором, и правообладателем созданного произведения, что позволяет ему по
собственному усмотрению использовать эту программу (или базу данных) в личных целях, продавать, раздавать
бесплатно, разрешать тиражировать и распространять или иным образом распоряжаться своими исключительными правами.

Кроме личных (неимущественных) прав автору “служебной” программы для ЭВМ (базы данных) принадлежит
право на вознаграждение при условии использования работодателем созданных произведений или передачи
исключительного права другому лицу. Размер и порядок выплаты этого определяется договором~[2,~ст.~1295,~п.~2].

\section{Нарушение прав на программу для ЭВМ и базу данных}

Специфика программ для ЭВМ такова, что они очень уязвимы в смысле их незаконного использования
(прежде всего, путем копирования и распространения копий). Незаконно изготовленные (скопированные)
или используемые экземпляры программы для ЭВМ или базы данных называются контрафактными, а
несанкционированное использование чужих программ или баз данных путем опубликования (выпуска в свет),
воспроизведения (полного или частичного), распространения, иного использования считается нарушением
исключительных прав на программы для ЭВМ или базы данных, т. е. нарушением авторского права.

\section{Право на официальную регистрацию}

В ст.~1262~ГК~РФ~Ч.4 закреплено право автора или иного правообладателя на государственную регистрацию
программы для ЭВМ или базы данных: <<Правообладатель в течение срока действия исключительного права
на программу для ЭВМ может по своему желанию зарегистрировать такую программу для ЭВМ или такую базу данных
в федеральном органе исполнительной власти по интеллектуальной собственности>>.
Исключение составляют программы для ЭВМ и базы данных, в которых содержатся сведения, составляющие
государственную тайну.

Предусмотренная регистрация не является правообразующей и носит факультативный характер,
т. е. с ней не связано возникновение прав на программу для ЭВМ, однако такая процедура представляется
полезной по следующим соображениям.

\begin{enumerate}
\item Она является официальным уведомлением общественности о наличии у правообладателей прав в
отношении рассматриваемых объектов.
\item Государственная регистрация содействует защите прав в случае возникновения конфликтных ситуаций
при нарушении прав или установлении приоритета.
\end{enumerate}

\section{Процедура официальной регистрации}

Процедура официальной регистрации программ для ЭВМ и баз данных в целом определена ст.~1262~ГК~РФ~Ч.4 и
включает подачу заявки в федеральный орган исполнительной власти по интеллектуальной собственности (Роспатент),
проверку поданных документов и собственно регистрацию. После поступления заявки на регистрацию в Роспатент
проверяется наличие необходимых документов и их соответствие установленным требованиям. При положительном
результате проверки сведения о программе для ЭВМ или базе данных вносятся, соответственно, в Реестр программ
для ЭВМ или Реестр баз данных под уникальным регистрационным номером и выдается заявителю (здесь заявителем
называют правообладателя, подавшего заявку на регистрацию программы или базы данных в Роспатент) свидетельство
о государственной регистрации установленной формы, в котором указаны регистрационный номер объекта по Реестру,
название программы или базы данных, имя или наименование правообладателя, фамилии авторов и дата регистрации.
Сведения о зарегистрированных программах для ЭВМ и базах данных публикуются в официальном бюллетене Роспатента.

\section{Заявка на официальную регистрацию}

Состав заявки на официальную регистрацию программы для ЭВМ или базы данных (далее - Заявка) определен
п.~2~ст.~1262~ГК~РФ~Ч.1, а также в Правилах составления, подачи и рассмотрения заявок на официальную
регистрацию программ для электронных вычислительных машин и баз данных (далее - Правила).

Заявка должна относиться к одной программе или одной базе данных.
При этом <<Программа для ЭВМ, состоящая из нескольких программ для ЭВМ (программный комплекс),
которые не могут быть использованы самостоятельно, регистрируется в целом (без регистрации каждой
входящей в нее (него) программы для ЭВМ)>>~[3,~п.~5]. Заявка должна содержать следующие документы:
\begin{enumerate}
\item заявление о государственной регистрации;
\item депонируемые материалы, идентифицирующие программу для ЭВМ, включая реферат;
\item документ, подтверждающий уплату государственной пошлины в установленном размере или
основание для освобождения от уплаты государственной пошлины или уменьшения его размера.
\end{enumerate}

В Правилах подробно описаны требования, предъявляемые к документам заявки.

Заявление на официальную регистрацию представляется отпечатанным на типографском бланке или в виде
компьютерной распечатки согласно образцам, приведенным в приложениях к Правилам (формы РП и РП/ДОП).

В состав депонируемых материалов входит также реферат, который представляется в двух экземплярах отдельно
от листинга программы для ЭВМ или описания структуры базы данных и не входит в их объем.
Реферат должен содержать информацию, определенную в п.п.~18а)~---~18и),~п.21~и~п.23~Правил,
в полном объеме. При этом:
\begin{enumerate}
\item аннотация реферата должна содержать сведения, определенные п.~18г)~Правил;
\item объем памяти указывается в Кбайтах или Мбайтах и определяется для программ как объем памяти,
занимаемый исходным текстом программы (листингом).
\end{enumerate}

\section{Программный продукт и формы его продажи}

Программный продукт --- персонифицированная программа для ЭВМ или база данных, которая предназначена
для самостоятельного использования конкретным пользователем в личных целях.

Коммерческая реализация (продажа) программного продукта связана с понятием использования программы для
ЭВМ третьими лицами (пользователями) и осуществляется на основании лицензионного договора с правообладателем.
Договор заключается в письменном виде и может определять следующие условия: способы использования, порядок
выплаты вознаграждения и срок действия договора, а также территорию, на которой используется данный
продукт~[2,~ст.~1235,~1236].

Одним из типов лицензионного договора на программу для ЭВМ является традиционный двухсторонний договор
правообладателя –-- лицензиара, с покупателем (пользователем) --- лицензиатом, в котором определяется
способы, сроки, территория использования программы или базы данных. Такие договоры составляются, как
правило, при единичных продажах программного продукта, предназначенного для решения достаточно узких
прикладных задач (научных, отраслевых и т. п.), при продажах программного продукта, требующего регулярного
обновления и дополнения (некоторые базы данных), а также при передаче прав на тиражирование и
распространение программ для ЭВМ или баз данных.

\section{Договор на использование программы для ЭВМ}

Текст договора должен содержать определенную исчерпывающую формулировку лицензионного соглашения между
владельцем прав на программу для ЭВМ (далее --- объект договора) и покупателем (приобретателем прав
на использование объекта договора).

\backmatter %% Здесь заканчивается нумерованная часть документа и начинаются ссылки и
            %% заключение

\Conclusion

В рамках дипломного проекта был проведён обзор последних тенденций на рынке Web приложений, который
показал востребованность технологии RIA, в частности Flex приложений. Были сформулирваны основные проблемы,
возникающие при тестировании взаимодействия Flex приложений с сервером, на основании которых
поставлено техническое задание. Распространённость технологии
Flex и неотъемлимость фазы тестирования в жизненном цикле программного обеспечения подтверждают
актуальность темы дипломного проекта.

Анализ уже существующих решений по тестированию Flex приложений показал,что ни одно из них в полной мере не
удовлетворяет свормулированным техническим требованиям, в результате чего было принято решение о разработке
собственного программного обеспечения, способного решить поставленную задачу.

Результатом реализации задачи дипломного проекта является программный модуль для тестового фреймворка
Apache JMeter, предоставляющий возможность функционального и нагрузочного тестирования взаимодействия Flex
приложения с сервером через AMF. В качестве серверной технологии используется BlazeDS.

Были проведены работы по модульному и функциональному тестированию разработанного ПО, показавшие его соответствие
заявленным требованиям.

Также в дипломном проекте предоставляется методика функционального и нагрузочного тестирования Flex приложений с использованием разработанного модуля,
демонстрирующая практическое применение его функционала.

В завершении дипломного проекта было составлено экономическое обоснование и приведен раздел,
посвященный защите интеллектуальной собственности.

Планируется дальнейшее развитие программного модуля с целью улучшения визуализации результатов тестов,
доработки пользовательского интерфейса и поддержки других функций BlazeDS. Исходный код приложения выложен в открытый
доступ на GitHub --- веб-сервис для хостинга проектов и их совместной разработки.



% % Список литературы при помощи BibTeX
% Юзать так:
%
% pdflatex diplom
% bibtex diplom
% pdflatex diplom

\bibliographystyle{gost780u}
\bibliography{diplom}


\appendix   % Тут идут приложения

\chapter{Программный код модуля}
\label{cha:appendix1}

\lstinputlisting[caption=Код реализация декодера сообщений]{../../src/main/java/edu/leti/amf/MessageDecoder.java}

\lstinputlisting[caption=Интерфейс обработки http-запросов]{../../src/main/java/edu/leti/jmeter/proxy/SamplerDeliverer.java}

\lstinputlisting[caption=Интерфейс прокси сервера]{../../src/main/java/edu/leti/jmeter/proxy/ProxyInterface.java}


\end{document}