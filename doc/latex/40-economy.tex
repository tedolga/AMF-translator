\chapter{Технико-экономическое обоснование}

\section{Концепция экономического обоснования}

На сегодняшний день одним из наиболее перспективных направлений в разработке
web-приложений является концепция Rich Internet Application (в дальнейшем RIA) --- это приложения,
доступные через сеть Интернет, обладающие особенностями и функциональностью традиционных настольных приложений.
Одной из наиболее распространенных технологий разработки RIA является Adobe Flex.
Flex приложения предоставляют возможность реализации клиент-серверного взаимодействия на основе бинарного формата
обмена данными --- AMF(Action Message Format). AMF более экономичен по трафику по сравнению с XML и позволяет
передавать типизированные объекты.
Как известно огромную роль в жизненном цикле программного обеспечения играет фаза тестирования.
Автоматизированное тестирование является его составной частью. Оно использует программные средства для
выполнения тестов и проверки их результатов, что помогает сократить время тестирования и упростить его процесс,
а также может дать возможность выполнять определенные тестовые задачи намного быстрее и эффективнее чем это может
быть сделано вручную. Однако использование AMF вызывает ряд трудностей для реализации автоматизации
функционального и нагрузочного тестирования взаимодействия сервера и Flex клиента, связанных с
бинарной природой протокола. Так как amf сообщение представляет собой совокупность байтов, разработчикам
и тестировщикам трудно считывать и изменять содержащуюся в нём информацию. Отдельной проблемой является
нагрузочное тестирование таких приложений --- имитация работы с приложением большого количества пользователей
за счёт запуска набора тестов в несколько потоков. Выходом из сложившейся ситуации является разработка
программного обеспечения, в значительной степени снижающего сложность осуществления функционального и
нагрузочного тестирования взаимодействия клиента и сервера по AMF-протоколу, а также разработка
методики функционального и нагрузочного тестирования использованием предложенного ПО.

Целью технико-экономического обоснования является определение экономической целесообразности реализации проекта.
Этапами ТЭО являются:

\begin{enumerate}
\item трудоемкость и календарный план выполнения НИР;
\item смета затрат на проведение НИР;
\item комплексная оценка эффективности НИР.
\end{enumerate}

\section{Трудоемкость и календарный план выполнения НИР}

\begin{landscape}
\begin{longtable}{|c|c|c|c|c|c|c|c|c|c|c|c|c|c|c|c|c|c|c|c|c|c|c|c|c|c|c|c|c|c|}
\caption{Трудоемкость и календарный план выполнения НИР}
\label{tab:longtable}
\\ \hline
\multirow{2}{*}{№}&\multirow{2}{*}{Этапы и работы}&\multicolumn{2}{|c|}{Трудоемкость, чел.-дн.}&\multirow{2}{*}{Численность, чел.}&\multirow{2}{*}{Длительность, дн.}&\multicolumn{24}{|c|}{Продолжительность работы (пятидневка)}\\\cline{3-4}\cline{7-30}
&&Исполнитель&Руководитель НИР&&&4&9&14&19&24&29&33&36&41&46&47&52&53&58&59&64&69&74&78&83&86&91&111&113\\\cline{3-4}\cline{7-30}
\hline \endfirsthead
\subcaption{Продолжение таблицы~\ref{tab:longtable}}
\\ \hline \endhead
\hline \subcaption{Продолжение на след. стр.}
\endfoot
\hline \endlastfoot
1& • & • & • & • & • & • & • & • & • & • & • & • & • & • & • & • & • & • & • & • & • & • & • & • & • & • & • & • & • \\
\hline 
2& • & • & • & • & • & • & • & • & • & • & • & • & • & • & • & • & • & • & • & • & • & • & • & • & • & • & • & • & • \\
\hline 
3 & • & • & • & • & • & • & • & • & • & • & • & • & • & • & • & • & • & • & • & • & • & • & • & • & • & • & • & • & • \\
\hline 
4 & • & • & • & • & • & • & • & • & • & • & • & • & • & • & • & • & • & • & • & • & • & • & • & • & • & • & • & • & • \\
\hline 
5 & • & • & • & • & • & • & • & • & • & • & • & • & • & • & • & • & • & • & • & • & • & • & • & • & • & • & • & • & • \\
\hline 
6 & • & • & • & • & • & • & • & • & • & • & • & • & • & • & • & • & • & • & • & • & • & • & • & • & • & • & • & • & • \\
\hline 
7 & • & • & • & • & • & • & • & • & • & • & • & • & • & • & • & • & • & • & • & • & • & • & • & • & • & • & • & • & • \\
\hline 
8 & • & • & • & • & • & • & • & • & • & • & • & • & • & • & • & • & • & • & • & • & • & • & • & • & • & • & • & • & • \\
\hline 
9 & • & • & • & • & • & • & • & • & • & • & • & • & • & • & • & • & • & • & • & • & • & • & • & • & • & • & • & • & • \\
\hline 
10 & • & • & • & • & • & • & • & • & • & • & • & • & • & • & • & • & • & • & • & • & • & • & • & • & • & • & • & • & • \\
\hline 
11 & • & • & • & • & • & • & • & • & • & • & • & • & • & • & • & • & • & • & • & • & • & • & • & • & • & • & • & • & • \\
\hline 
12 & • & • & • & • & • & • & • & • & • & • & • & • & • & • & • & • & • & • & • & • & • & • & • & • & • & • & • & • & • \\
\hline 
13 & • & • & • & • & • & • & • & • & • & • & • & • & • & • & • & • & • & • & • & • & • & • & • & • & • & • & • & • & • \\
\hline 
14 & • & • & • & • & • & • & • & • & • & • & • & • & • & • & • & • & • & • & • & • & • & • & • & • & • & • & • & • & • \\
\hline 
15 & • & • & • & • & • & • & • & • & • & • & • & • & • & • & • & • & • & • & • & • & • & • & • & • & • & • & • & • & • \\
\hline 
16 & • & • & • & • & • & • & • & • & • & • & • & • & • & • & • & • & • & • & • & • & • & • & • & • & • & • & • & • & • \\
\hline 
16 & • & • & • & • & • & • & • & • & • & • & • & • & • & • & • & • & • & • & • & • & • & • & • & • & • & • & • & • & • \\
\hline
16 & • & • & • & • & • & • & • & • & • & • & • & • & • & • & • & • & • & • & • & • & • & • & • & • & • & • & • & • & • \\
\hline
16 & • & • & • & • & • & • & • & • & • & • & • & • & • & • & • & • & • & • & • & • & • & • & • & • & • & • & • & • & • \\
\hline
16 & • & • & • & • & • & • & • & • & • & • & • & • & • & • & • & • & • & • & • & • & • & • & • & • & • & • & • & • & • \\
\hline
16 & • & • & • & • & • & • & • & • & • & • & • & • & • & • & • & • & • & • & • & • & • & • & • & • & • & • & • & • & • \\
\hline
16 & • & • & • & • & • & • & • & • & • & • & • & • & • & • & • & • & • & • & • & • & • & • & • & • & • & • & • & • & • \\
\hline
\end{longtable}
\end{landscape}

Трудоемкость выполнения работы исполнителем  составляет 106 чел.-дней, а руковолителем 7 чел.-дней.
Общая продолжительность выполнения данной НИР 113 дней (чуть больше 15 недель).

\section{Cмета затрат на проведение НИР}

Проводится расчет затрат, связанных с проведением НИР. Основные статьи калькуляции приведены в табл. 4.1.

