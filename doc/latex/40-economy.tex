\chapter{Технико-экономическое обоснование}

\section{Концепция экономического обоснования}

На сегодняшний день одним из наиболее перспективных направлений в разработке
web-приложений является концепция Rich Internet Application (в дальнейшем RIA) --- это приложения,
доступные через сеть Интернет, обладающие особенностями и функциональностью традиционных настольных приложений.
Одной из наиболее распространенных технологий разработки RIA является Adobe Flex.
Flex приложения предоставляют возможность реализации клиент-серверного взаимодействия на основе бинарного формата
обмена данными --- AMF(Action Message Format). AMF более экономичен по трафику по сравнению с XML и позволяет
передавать типизированные объекты.
Как известно огромную роль в жизненном цикле программного обеспечения играет фаза тестирования.
Автоматизированное тестирование является его составной частью. Оно использует программные средства для
выполнения тестов и проверки их результатов, что помогает сократить время тестирования и упростить его процесс,
а также может дать возможность выполнять определенные тестовые задачи намного быстрее и эффективнее чем это может
быть сделано вручную. Однако использование AMF вызывает ряд трудностей для реализации автоматизации
функционального и нагрузочного тестирования взаимодействия сервера и Flex клиента, связанных с
бинарной природой протокола. Так как amf сообщение представляет собой совокупность байтов, разработчикам
и тестировщикам трудно считывать и изменять содержащуюся в нём информацию. Отдельной проблемой является
нагрузочное тестирование таких приложений --- имитация работы с приложением большого количества пользователей
за счёт запуска набора тестов в несколько потоков. Выходом из сложившейся ситуации является разработка
программного обеспечения, в значительной степени снижающего сложность осуществления функционального и
нагрузочного тестирования взаимодействия клиента и сервера по AMF-протоколу, а также разработка
методики функционального и нагрузочного тестирования использованием предложенного ПО.

Целью технико-экономического обоснования является определение экономической целесообразности реализации проекта.
Этапами ТЭО являются:

\begin{enumerate}
\item трудоемкость и календарный план выполнения НИР;
\item смета затрат на проведение НИР;
\item комплексная оценка эффективности НИР.
\end{enumerate}

\section{Рынок и план маркетинга}

\subsection{Сегментирование рынка}

Сегментирование рынка состоит в выделении сегментов анализируемого рынка и оценки спроса
на продукцию в каждом сегменте рынка.

Итак, выделим основные сегменты:
\begin{enumerate}
\item Компании, осуществляющие разработку программного обеспечения.
\item Компании, специализирующиеся на внешнем тестировании программного обеспечения.
\item Программисты, осуществляющие разработку ПО в частном порядке (для личных или коммерческих целей).
\end{enumerate}

Проведем анализ требований различных групп потенциальных покупателей к продукции:
\begin{enumerate}
\item технический уровень, качество и надежность в эксплуатации, уровень послепродажного обслуживания;
\item технический уровень, качество и надежность в эксплуатации, уровень послепродажного обслуживания;
\item цена продукции, качество и надежность в эксплуатации.
\end{enumerate}

\subsection{Продвижение товара}
Для эффективного продвижения товара, а так же для поддержания спроса необходима реклама, предоставление скидок,
 демо-версии для привлечения новых клиентов. Поскольку продукт является специализированным, следует давать 
 рекламу в журналы, газеты, размещать на сайтах и форумах той же тематики. Скидки, как правило, стимулируют 
 уже существующих клиентов, поскольку важно, привлекая новых, не потерять уже существующих пользователей.

\section{Производство продукта}

Проводится расчет затрат, связанных с проведением НИР. Основные статьи калькуляции приведены ниже:
\begin{enumerate}
\item Материалы
\item Спецоборудование
\item Расходы на оплату труда
\item Отчисления на социальные нужды
\item Затраты по работам, выполняемым сторонними организациями
\item Командировочные расходы
\item Прочие прямые расходы
\item Накладные расходы
\end{enumerate}

\subsection{Статья <<Материалы>>}

На статью относятся затраты на сырье, основные и вспомогательные материалы, покупные  полуфабрикаты
и комплектующие изделия, необходимые для выполнения конкретной НИР c учетом транспортно-заготовительных
расходов. Калькуляция расходов по статье «Материалы» приведена в табл.~\ref{tab:material}.

\begin{table}[ht]
\caption{Материалы}
\begin{tabular}{|c|c|c|c|}
\hline
Наименование материалов&Коли-чество, шт&Цена, р.&Сумма, р.\\
\hline
Бумага формата А4&1&120&120\\
\hline
Заправка картриджа для принтера&1&300&300\\
\hline
Записываемый диск CD-RW&1&50&50\\
\hline
\multicolumn{3}{|r|}{Итого:}&470\\
\hline
\multicolumn{3}{|r|}{Транспортные расходы, 15\%:}&70\\
\hline
\multicolumn{3}{|r|}{Всего:}&540\\
\hline
\end{tabular}
\label{tab:material}
\end{table}

\subsection{Статья <<Спецоборудование>>}

На статью <<Спецоборудование>> относятся затраты на приобретение (или изготовление) специальных приборов,
стендов, другого специального оборудования, необходимого для выполнения конкретной НИР.
Для данной НИР дополнительного спецоборудования не требуется.

\subsection{Статья <<Расходы на оплату труда>>}

На статью относится заработная плата научных сотрудников, инженеров и прочего инженерно-технического персонала,
непосредственно занятых выполнением конкретной НИР. Эти расходы складываются из основной и дополнительной
заработной платы. Средняя зарплата за один рабочий день определяется, исходя из месячного оклада и среднего
количества рабочих дней за месяц, принимаемого за 22 дня.\\
Основная зарплата рассчитывается по формуле:

\begin{equation}
C_{зо} = \frac{T \cdot C_{зо.мес}}{22}\mbox{,}
\label{F:F1}
\end{equation}

где Т – трудоемкость выполнения работ по НИР, $С_{зо.мес}$ – месячный оклад.\\
Дополнительная зарплата рассчитывается по формуле:

\begin{equation}
C_{зд} = \frac{C_{зо} \cdot Н_{д}}{100}\mbox{,}
\label{F:F2}
\end{equation}

где $C_{зо}$ – основная зарплата, $Н_{д}$ – норматив дополнительной зарплаты.\\
Производим расчет основной и дополнительной зарплат на основании следующих данных:

\begin{enumerate}
\item трудоемкость выполнения работ исполнителем $Т_{исп}$ = 106~чел.-дн.;
\item трудоемкость выполнения работ СНС $Т_{рук}$= 7~чел.-дн.;
\item месячный оклад инженера $С_{зо.мес.исп}$ = 15~000~р.;
\item месячный оклад руководителя $С_{зо.мес.рук}$ = 20~000~р.;
\item норматив дополнительной зарплаты $Н_{д}$ = 12\%.
\end{enumerate}

%\begin{equation}
\begin{align}
&С_{зо.исп} = \frac{106 \cdot 15 000}{22} = 72 272 р.\mbox{,}\\
&С_{зо.рук} = \frac{7 \cdot 20 000}{22} = 6 363 р.\mbox{,}\\
&С_{зо} = 72 272 + 6 363 = 78 635 р.\mbox{,}\\
&С_{зд.исп} = \frac{72 272 \cdot 12}{100} = 8 672 р.\mbox{,}\\
&С_{зд.рук} = \frac{6 363 \cdot 12}{100} = 763 р.\mbox{,}\\
&С_{зд} = 8 672 + 763 = 9 435 р.
\end{align}
%\end{equation}

Общие расходы на оплату труда составляют 88 070 р.

\subsection{Статья <<Отчисления на социальные нужды>>}

На статью относят затраты, связанные с выплатой единого социального налога.
Данная статья рассчитывается пропорционально зарплате в размере 26,2\%:
\begin{enumerate}
\item федеральный бюджет – 20\%;
\item фонд социального страхования – 3.2\%;
\item фонд обязательного медицинского страхования – 2,8\%;
\item страхование от несчастных случаев – 0,2\%.
\end{enumerate}

\begin{equation}
C_{зд} = \frac{P_{от} \cdot Н_{сн}}{100}\mbox{,}
\label{F:F3}
\end{equation}

где $Н_{сн}$ – суммарный норматив отчислений на социальные нужды – 26,2%.

\begin{equation}
С_{сн} = \frac{88070 \cdot 26,2}{100} = 23 074 р.\mbox{,}
\label{F:F4}
\end{equation}

\subsection{Статья <<Затраты  по работам, выполняемым сторонними организациями>>}

Расходов на работы, выполняемые сторонними организациями, не существует.

\subsection{Статья <<Командировочные расходы>>}

На статью относятся расходы на все виды служебных командировок работников, выполняющих
задания по конкретной НИР. В данной НИР расходов, связанных со служебными командировками, нет.

\subsection{Статья <<Прочие прямые расходы>>}

На статью относятся расходы на получение специальной научно-технической информации, платежи
за использование средств связи и коммуникации, а также другие расходы, необходимые для проведения НИР.
На всех этапах работы требуется выход в Интернет. Стоимость работы в Интернете составляет 350~р. в месяц.
Затраты на Интернет на весь период составляют:

\begin{equation}
С_{и} = \frac{350}{22} \cdot 106 = 1 686 р.\mbox{,}
\label{F:F5}
\end{equation}

Также требуется использование телефона. Примем эти расходы ориентировочно равными $С_{т}$ = 300 р.
Следовательно, суммарно прочие прямые затраты составляют:

\begin{equation}
С_{ппз} = С_{и} + С_{т} = 1 686 + 300 = 1 986 р.\mbox{,}
\label{F:F6}
\end{equation}

\subsection{Статья <<Накладные расходы>>}

В статью включаются расходы на управление и хозяйственное обслуживание НИР.

\begin{equation}
С_{нр} = \frac{P_{от} \cdot H_{нр}}{100}\mbox{,}
\label{F:F5}
\end{equation}

где $Н_{нр}$ – норма накладных расходов, равная 20\%.

\begin{equation}
С_{нр} = 88 070 \cdot \frac{20}{100} = 17 614 р.
\label{F:F5}
\end{equation}

На основании полученных данных в табл.~\ref{tab:calc} приведена калькуляция себестоимости разработки.

\begin{table}[ht]
\caption{Смета затрат на проведение НИР}
\begin{tabular}{|c|p{10cm}|c|}
\hline
№ п/п&Наименование статьи&Сумма, р.\\
\hline
1&Материалы&540\\
\hline
2&Спецоборудование&---\\
\hline
3&Расходы на оплату труда&88070\\
\hline
4&Отчисления на социальные нужды&23074\\
\hline
5&Затраты по работам, выполняемым сторонними организациями&---\\
\hline
6&Командировочные расходы&---\\
\hline
7&Прочие прямые расходы&1986\\
\hline
8&Накладные расходы&17614\\
\hline
\multicolumn{2}{|r|}{Себестоимость НИР:}&131 284\\
\hline
\end{tabular}
\label{tab:calc}
\end{table}

\section{Организационный план проекта}

Трудоемкость выполнения работы исполнителем  составляет 106 чел.-дней, а руковолителем 7 чел.-дней.
Общая продолжительность выполнения данной НИР 113 дней (чуть больше 15 недель).

\section{Комплексная оценка эффективности НИР}

\subsection{Научно-технический эффект разработки}

Концепции и технологии, используемые при разработке web-приложений,
постоянно развиваются и совершенствуются - оптимизируется использование ресурсов и времени,
улучшаются возможности по отображению предоставляемой информации, динамичность и их интерактивность.

Flex является одним из самых распространённых и перспективных инструментов разработки web-приложений.
Разрабатываемый в дипломном проекте программный модуль поможет обеспечить контроль качества Flex
приложений, а также повысит эффективность тестирования за счёт автоматизации тестовых сценариев.

\subsection{Экономический эффект}

Выгоды, которые может получить потребитель от использования разрабатываемой продукции:

\begin{enumerate}
\item Повышение стабильности разрабатываемого приложения за счёт обеспечения качественного тестирования.
\item Повышение скорости тестирования за счёт автоматизации труда инженеров по качеству.
\item Понижение затрат на тестирование, т.к. внедрение разрабатываемого модуля позволяет отказаться от
использования платных средств автоматизированного тестирования.
\end{enumerate}

\subsection{Расчет потребности в начальных инвестициях}

Потребность в начальном капитале определяется средствами, израсходованных на НИР – 131 284 р.,
a также дополнительными средствами, необходимых для приобретения ПЭВМ и принтера. При стоимости ПЭВМ 20 000 р.,
сроке ее службы – 5 лет, времени копирования программного обеспечения 0,5 ч. потребность в ПЭВМ составит: