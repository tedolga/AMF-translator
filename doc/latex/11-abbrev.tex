\Abbreviations %% Список обозначений и сокращений в тексте
\begin{description}
\item[RIA] Rich Internet application, приложения, доступные через сеть Интернет, обладающие особенностями и
функциональностью традиционных настольных приложений.
\item[AMF] Action Message Format, бинарный формат обмена данными.
\item[ПО] программное обеспечение.
\item[GUI] Graphical user interface, графический интерфейс пользователя.
\item[URL] Uniform Resource Locator, единообразный локатор ресурса.
\item[HTTP] HyperText Transfer Protocol,  протокол прикладного уровня передачи данных (изначально — в виде
гипертекстовых документов).
\item[MW] Microsoft Windows, семейство проприетарных операционных систем корпорации Майкрософт (Microsoft).
\item[CP] Cross-Platform, программное обеспечение, работающее более чем на одной аппаратной платформе и/или
операционной системе.
\item[PS] Proprietary Software,  программное обеспечение, являющееся частной собственностью авторов или правообладателей
 и не удовлетворяющее критериям свободного ПО
\item[APL] Apache License,  лицензия на свободное программное обеспечение Apache Software Foundation.
\item[MVC] Model-view-controller, схема использования нескольких шаблонов проектирования, с помощью которых модель
данных приложения, пользовательский интерфейс и взаимодействие с пользователем разделены на три отдельных компонента.
\end{description}
