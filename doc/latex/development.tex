\chapter*{Разработка модуля}
\addcontentsline{toc}{chapter}{Разработка модуля}

\section*{Анализ требований к модулю}
\addcontentsline{toc}{section}{Анализ требований к модулю}

\subsection*{Требования}
\addcontentsline{toc}{subsection}{Требования}
На основании приведённых ранее целей и задач проекта были разработаны функциональны требования.
Для наглядности и удобства они были сведены в таблицу.
Для классификации требований были выбраны следующие критерии:
\begin{enumerate}
\item Приоритет --- критерий оценки полезности реализации требования для конечного пользователя,
важности для достижения поставленных перед проектом целей.
\item Трудоёмкость --- критерий отображает предварительную оценку сложности реализации требования,
количество привлекаемых для этого ресурсов.
\item Риск --- интегральный критерий, введением которого предпринимается попытка оценить во-первых
возможность невыполнения требования, во-вторых возможную ошибку в оценке по двум предыдущим
критериям (в первую очередь  трудоёмкости).
\end{enumerate}

Для каждого из критериев введена шкала из трёх уровней: низкий, средний, высокий.
Размерность шкалы выбрана минимальной исходя из соображений простоты и наглядности.
Так как размер проекта и количество предъявляемых к нему требований незначительны,
то таких шкал вполне достаточно для исчерпывающей классификации требований а также принятия
проектных и организационных решений.

\section*{Выбор технического средства решения задачи}
\addcontentsline{toc}{section}{Выбор технического средства решения задачи}

\section*{Архитектура модуля}
\addcontentsline{toc}{section}{Архитектура модуля}

\section*{Реализация}
\addcontentsline{toc}{section}{Реализация}

\section*{Методика}
\addcontentsline{toc}{section}{Методика}
