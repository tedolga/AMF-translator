\begin{abstract}
Пояснительная записка 102 с., 27 рис., 11 табл., 12 источников.

Целью настоящего дипломного проекта является разработка программного обеспечения, предоставляющего возможность
проведения тестирования взаимодействия Flex приложений с сервером через AMF протокол.

Полученные результаты: разработан программный модуль для тестового фреймворка Apache JMeter, предоставляющий
возможность проведения функционального и нагрузочного тестирования взаимодействия Flex приложений с сервером через AMF
протокол. Тестирование модуля показало его соответствие техническим требованиям. Также в дипломном проекте
предоставляется методика тестирования Flex приложений с использованием разработанного модуля, демонстрирующая
практическое применение его функционала.

Преимуществами разработанного программного обеспечения являются:

\begin{enumerate}
\item возможность записи тестовых сценариев с помощью прокси сервера.
\item кросплатформенность;
\item расширяемость;
\end{enumerate}

Разработанный программный комплекс может быть использован программистами и инженерами по тестированию в различных
организациях, принадлежащих сфере IT, в целях повышения качества разрабатываемого программного обеспечения.
\end{abstract}
