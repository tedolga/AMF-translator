\Defines
\begin{description}
\item[Фреймворк] в информационных системах структура программной системы; программное обеспечение, облегчающее
разработку и объединение разных компонентов большого программного проекта.
\item[Apache JMeter] инструмент для тестирования программного обеспечения, разрабатываемый Apache Jakarta Project.
\item[Java] объектно-ориентированный язык программирования, разработанный компанией Sun Microsystems
(в последующем, приобретённой компанией Oracle).
\item[BlazeDS] серверная Java-технология для передачи данных.
\item[Flex] технология для создания RIA, разработанная компанией Adobe.
\item[Maven] средство для автоматизации сборки проектов.
\item[Прокси-сервер] служба (комплекс программ) в компьютерных сетях, позволяющая клиентам выполнять косвенные
запросы к другим сетевым службам.
\item[Проприетарное программное обеспечение] программное обеспечение, являющееся частной собственностью авторов или
правообладателей и не удовлетворяющее критериям свободного ПО.
\end{description}
