\chapter{Охрана интеллектуальной собственности}

\section{Интеллектуальная собственность}

Согласно определению интеллектуальной собственности, принятому в российском законодательстве,
а также на основании определения Стокгольмской конференции от 14 июля 1967 г., компьютерные программы
относятся к объектам интеллектуальной собственности. Компьютерным программам предоставляется охрана
нормами авторского права как литературным произведениям в соответствии с Бернской конвенцией.
В Российской Федерации вопросы предоставления правовой охраны программам регулируются Гражданским
кодексом РФ, Часть~4~(ГК~РФ~Ч.4).

\section{Программа для ЭВМ}

Под программой для ЭВМ понимается <<... представленная в объективной форме совокупность данных и команд,
предназначенных для функционирования ЭВМ и других компьютерных устройств в целях получения определенного
результата>>. Кроме того, в понятие программы для ЭВМ входят <<...подготовительные материалы, полученные
в ходе разработки программы для ЭВМ, и порождаемые ею аудиовизуальные отображения>>~[2,~ст.~1261].
С точки зрения программистов и пользователей программа для ЭВМ представляет собой детализацию алгоритма
решения какой-либо задачи и выражена в форме определенной последовательности предписаний, обеспечивающих
выполнение компьютером преобразования исходных данных в искомый результат.

Можно выделить следующие объективные формы представления программы для ЭВМ:
\begin{enumerate}
\item исходная программа (или исходный текст) --- последовательность предписаний на алгоритмическом языке
высокого уровня, предназначенных для автоматизированного перевода этих предписаний в последовательность команд
в объектном коде;
\item рабочая программа (или объектный код) --- последовательность машинных команд, т. е. команд, представленных
на языке, понятном ЭВМ;
\item программа, временно введенная в память ЭВМ, --- совокупность физических состояний элементов памяти
запоминающего устройства ЭВМ (ОЗУ), сохраняющихся до прекращения подачи электропитания к ЭВМ;
\item программа, постоянно хранимая в памяти ЭВМ, --- представленная на языке машины команда (или серия команд),
выполненная в виде физических особенностей участка интегральной схемы, сохраняющихся независимо от подачи
электропитания.
\end{enumerate}

Исходная и рабочая программы представляются в электронном виде.
Правовая охрана программ для ЭВМ распространяется только в отношении формы их выражения и <<… не распространяется
на идеи, концепции, принципы, методы, процессы, системы, способы, решения технических, организационных или иных
задач, открытия, факты, языки программирования>>~[2,~ст.1259,~п.~5].

\section{Авторское право на программу для ЭВМ}

Предпосылкой охраноспособности программы для ЭВМ и базы данных является их творческий характер, т. е. они
должны быть продуктом личного творчества автора. Творческий характер деятельности автора предполагается до
тех пор, пока не доказано обратное~[2,~ст.~1257].

Момент возникновения авторского права является важнейшим юридическим фактом, который устанавливается в силу
создания программы для ЭВМ. <<Для возникновения, осуществления и защиты авторских прав не требуется регистрация
произведения или соблюдение каких-либо иных формальностей>>~[2,~ст.1259,~п.4].

Закон устанавливает, что обнародование программы не является обязательным условием для возникновения прав на нее:
<<Авторские права распространяется как на обнародованные, так и на необнародованные произведения, выраженные в
какой-либо объективной форме ...>>~[2,~ст.~1259,~п.~3].

Таким образом, только сам факт создания программы, зафиксированной в объективной форме, является основанием
возникновения авторского права на нее.

Каждая составляющая понятия использования программы для ЭВМ имеет конкретное содержание, которое также
определено законом:
\begin{enumerate}
\item воспроизведение --- <<... изготовление одного или более экземпляров произведения или его части
в любой материальной форме, … в том числе запись в память ЭВМ>>~[2,~ст.~1270,~п.~2,~п.п.1];
\item распространение –-- предоставление доступа к произведению <<... путем продажи или иного отчуждения его
оригинала или экземпляров>>~[2,~ст.~1270,~п.~2,~п.п.2];
\item публичный показ --- <<... любая демонстрация оригинала или экземпляров произведения непосредственно ...
либо с помощью технических средств в месте, открытом для свободного посещения, или в месте, где присутствует
значительное число лиц ...>>~[2,~ст.~1270,~п.~2,~п.п.3].
\end{enumerate}

В целях оповещения о своих правах правообладатель <<... вправе использовать знак охраны авторского права,
который помещается на каждом экземпляре произведения и состоит из следующих элементов:
латинской буквы С в окружности; имени или наименования правообладателя; года первого опубликования
произведения>>~[2,~ст.~1271].

Исключительные права на программу переходят по наследству в установленном законом порядке, и их можно
реализовать в течение срока действия авторского права.

Передача прав на материальный носитель не влечет за собой передачи каких-либо прав на программу для
ЭВМ~[2,~ст.1227]. Иными словами, передача носителя информации (например, диска) с зафиксированной на
нем программой третьему лицу не означает передачи каких-либо прав на эту программу.

\section{Правообладание}

Если человек (или группа людей) самостоятельно, по личной инициативе создал программу для ЭВМ,
то он является одновременно и автором, и правообладателем созданного произведения, что позволяет ему по
собственному усмотрению использовать эту программу (или базу данных) в личных целях, продавать, раздавать
бесплатно, разрешать тиражировать и распространять или иным образом распоряжаться своими исключительными правами.

Кроме личных (неимущественных) прав автору “служебной” программы для ЭВМ (базы данных) принадлежит
право на вознаграждение при условии использования работодателем созданных произведений или передачи
исключительного права другому лицу. Размер и порядок выплаты этого определяется договором~[2,~ст.~1295,~п.~2].

\section{Нарушение прав на программу для ЭВМ и базу данных}

Специфика программ для ЭВМ такова, что они очень уязвимы в смысле их незаконного использования
(прежде всего, путем копирования и распространения копий). Незаконно изготовленные (скопированные)
или используемые экземпляры программы для ЭВМ или базы данных называются контрафактными, а
несанкционированное использование чужих программ или баз данных путем опубликования (выпуска в свет),
воспроизведения (полного или частичного), распространения, иного использования считается нарушением
исключительных прав на программы для ЭВМ или базы данных, т. е. нарушением авторского права.

\section{Право на официальную регистрацию}

В ст.~1262~ГК~РФ~Ч.4 закреплено право автора или иного правообладателя на государственную регистрацию
программы для ЭВМ или базы данных: <<Правообладатель в течение срока действия исключительного права
на программу для ЭВМ может по своему желанию зарегистрировать такую программу для ЭВМ или такую базу данных
в федеральном органе исполнительной власти по интеллектуальной собственности>>.
Исключение составляют программы для ЭВМ и базы данных, в которых содержатся сведения, составляющие
государственную тайну.

Предусмотренная регистрация не является правообразующей и носит факультативный характер,
т. е. с ней не связано возникновение прав на программу для ЭВМ, однако такая процедура представляется
полезной по следующим соображениям.

\begin{enumerate}
\item Она является официальным уведомлением общественности о наличии у правообладателей прав в
отношении рассматриваемых объектов.
\item Государственная регистрация содействует защите прав в случае возникновения конфликтных ситуаций
при нарушении прав или установлении приоритета.
\end{enumerate}

\section{Процедура официальной регистрации}

Процедура официальной регистрации программ для ЭВМ и баз данных в целом определена ст.~1262~ГК~РФ~Ч.4 и
включает подачу заявки в федеральный орган исполнительной власти по интеллектуальной собственности (Роспатент),
проверку поданных документов и собственно регистрацию. После поступления заявки на регистрацию в Роспатент
проверяется наличие необходимых документов и их соответствие установленным требованиям. При положительном
результате проверки сведения о программе для ЭВМ или базе данных вносятся, соответственно, в Реестр программ
для ЭВМ или Реестр баз данных под уникальным регистрационным номером и выдается заявителю (здесь заявителем
называют правообладателя, подавшего заявку на регистрацию программы или базы данных в Роспатент) свидетельство
о государственной регистрации установленной формы, в котором указаны регистрационный номер объекта по Реестру,
название программы или базы данных, имя или наименование правообладателя, фамилии авторов и дата регистрации.
Сведения о зарегистрированных программах для ЭВМ и базах данных публикуются в официальном бюллетене Роспатента.

\section{Заявка на официальную регистрацию}

Состав заявки на официальную регистрацию программы для ЭВМ или базы данных (далее - Заявка) определен
п.~2~ст.~1262~ГК~РФ~Ч.1, а также в Правилах составления, подачи и рассмотрения заявок на официальную
регистрацию программ для электронных вычислительных машин и баз данных (далее - Правила).

Заявка должна относиться к одной программе или одной базе данных.
При этом <<Программа для ЭВМ, состоящая из нескольких программ для ЭВМ (программный комплекс),
которые не могут быть использованы самостоятельно, регистрируется в целом (без регистрации каждой
входящей в нее (него) программы для ЭВМ)>>~[3,~п.~5]. Заявка должна содержать следующие документы:
\begin{enumerate}
\item заявление о государственной регистрации;
\item депонируемые материалы, идентифицирующие программу для ЭВМ, включая реферат;
\item документ, подтверждающий уплату государственной пошлины в установленном размере или
основание для освобождения от уплаты государственной пошлины или уменьшения его размера.
\end{enumerate}

В Правилах подробно описаны требования, предъявляемые к документам заявки.

Заявление на официальную регистрацию представляется отпечатанным на типографском бланке или в виде
компьютерной распечатки согласно образцам, приведенным в приложениях к Правилам (формы РП и РП/ДОП).

В состав депонируемых материалов входит также реферат, который представляется в двух экземплярах отдельно
от листинга программы для ЭВМ или описания структуры базы данных и не входит в их объем.
Реферат должен содержать информацию, определенную в п.п.~18а)~---~18и),~п.21~и~п.23~Правил,
в полном объеме. При этом:
\begin{enumerate}
\item аннотация реферата должна содержать сведения, определенные п.~18г)~Правил;
\item объем памяти указывается в Кбайтах или Мбайтах и определяется для программ как объем памяти,
занимаемый исходным текстом программы (листингом).
\end{enumerate}

\section{Программный продукт и формы его продажи}

Программный продукт --- персонифицированная программа для ЭВМ или база данных, которая предназначена
для самостоятельного использования конкретным пользователем в личных целях.

Коммерческая реализация (продажа) программного продукта связана с понятием использования программы для
ЭВМ третьими лицами (пользователями) и осуществляется на основании лицензионного договора с правообладателем.
Договор заключается в письменном виде и может определять следующие условия: способы использования, порядок
выплаты вознаграждения и срок действия договора, а также территорию, на которой используется данный
продукт~[2,~ст.~1235,~1236].

Одним из типов лицензионного договора на программу для ЭВМ является традиционный двухсторонний договор
правообладателя –-- лицензиара, с покупателем (пользователем) --- лицензиатом, в котором определяется
способы, сроки, территория использования программы или базы данных. Такие договоры составляются, как
правило, при единичных продажах программного продукта, предназначенного для решения достаточно узких
прикладных задач (научных, отраслевых и т. п.), при продажах программного продукта, требующего регулярного
обновления и дополнения (некоторые базы данных), а также при передаче прав на тиражирование и
распространение программ для ЭВМ или баз данных.

\section{Договор на использование программы для ЭВМ}

Текст договора должен содержать определенную исчерпывающую формулировку лицензионного соглашения между
владельцем прав на программу для ЭВМ (далее --- объект договора) и покупателем (приобретателем прав
на использование объекта договора).